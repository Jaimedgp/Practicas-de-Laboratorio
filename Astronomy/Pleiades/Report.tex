%----------------------------------------------------------------------------------------
%	PACKAGES AND OTHER DOCUMENT CONFIGURATIONS
%----------------------------------------------------------------------------------------
\documentclass[twoside]{article}

\usepackage[sc]{mathpazo} % Use the Palatino font
\usepackage[english]{babel}
\usepackage[utf8]{inputenc}
\usepackage{lipsum}
\usepackage{graphicx}
%\usepackage[T1]{fontenc} % Use 8-bit encoding that has 256 glyphs
\linespread{1.15} % Line spacing - Palatino needs more space between lines
\usepackage{microtype} % Slightly tweak font spacing for aesthetics

\usepackage[hmarginratio=1:1,top=32mm,columnsep=20pt]{geometry} % Document margins
\usepackage{multicol} % Used for the two-column layout of the document
\usepackage[hang, small,labelfont=bf,up,textfont=it,up]{caption} % Custom captions under/above floats in tables or figures
\usepackage{mathtools}
\usepackage{booktabs} % Horizontal rules in tables
\usepackage{float} % Required for tables and figures in the multi-column environment - they need to be placed in specific locations with the [H] (e.g. \begin{table}[H])
\usepackage{hyperref} % For hyperlinks in the PDF
\usepackage{wrapfig}
%\usepackage[]{mcode} % For embebing matlab code
\usepackage[makeroom]{cancel}

%\usepackage{lettrine} % The lettrine is the first enlarged letter at the beginning of the text
\usepackage{paralist} % Used for the compactitem environment which makes bullet points with less space between them

\usepackage{abstract} % Allows abstract customization
\renewcommand{\abstractnamefont}{\normalfont\bfseries} % Set the "Abstract" text to bold
\renewcommand{\abstracttextfont}{\normalfont\small\itshape} % Set the abstract itself to small italic text

\usepackage{titlesec} % Allows customization of titles
%\renewcommand\thesection{\Roman{section}} % Roman numerals for the sections
%\renewcommand\thesubsection{\Roman{subsection}} % Roman numerals for subsections
\titleformat{\section}[block]{\large\scshape\centering}{\thesection.}{1em}{} % Change the look of the section titles
\titleformat{\subsection}[block]{\large\centering}{\thesubsection.}{1em}{} % Change the look of the section titles

\usepackage{fancyhdr} % Headers and footers
\pagestyle{fancy} % All pages have headers and footers
\fancyhead{} % Blank out the default header
\fancyfoot{} % Blank out the default footer
\fancyhead[C]{Experimental Optics % based on TRACS 
\hspace{4pt} $\bullet$ \hspace{4pt} Short Title
\hspace{4pt} $\bullet$ \hspace{4pt} Month Year} % Custom header text
\fancyfoot[RO,LE]{\thepage} % Custom footer text

\usepackage{cite}

\DeclareGraphicsExtensions{.pdf,.png,.jpg} % Graphics type

%----------------------------------------------------------------------------------------
%	   TITLE SECTION
%----------------------------------------------------------------------------------------

\title{
	\vspace{-15mm}
	\fontsize{28pt}{10pt}
	\selectfont\textbf{TITLE}% Article title
}

\author{
	\large
	\textsc{Jaime Díez González-Pardo}\\[4mm]%\thanks{A thank you or further information}\\[2mm] % Your name
	\fontsize{28pt}{10pt} University of Cantabria \\ % Your institution
	%\thanks{A thank you or further information}\\[2mm] % Your name
	\normalsize Experimental Optics \\ 
	%\normalsize{Compañera:} \textsc{Mónica Escobedo}\\%\normalsize \href{mailto:john@smith.com}{john@smith.com} % Your email address
	%\vspace{5mm}
}

\date{ \today }


%----------------------------------------------------------------------------------------
%      · DOCUMENT
%----------------------------------------------------------------------------------------

\begin{document}


	\maketitle % Insert title


	\thispagestyle{fancy} % All pages have headers and footers

%----------------------------------------------------------------------------------------
%	  ABSTRACT
%----------------------------------------------------------------------------------------

	\begin{abstract}

		\noindent% Dummy abstract text

	\end{abstract}

%----------------------------------------------------------------------------------------
%	  ARTICLE CONTENTS
%----------------------------------------------------------------------------------------

	\begin{multicols}{2} % Two-column layout throughout the main article text

		\section{Introduction} % Scope of the project = rad effects + minimization

			

		\section{Methods}

			

		\section{Results}



		\section{Conclussions}

	\end{multicols}

%----------------------------------------------------------------------------------------
%     BIBLIOGRAPHY
%----------------------------------------------------------------------------------------

	%\bibliographystyle{unsrt}
	%\bibliography{biblio}

%----------------------------------------------------------------------------------------
%     APPENDIX
%----------------------------------------------------------------------------------------

%\newpage

		%\appendix

\end{document}

%--------------------------------------------------------------------------------------
%            TEMPLATES
%--------------------------------------------------------------------------------------

%----------------------------------------------------------------------------------------
%            how to insert an image
%----------------------------------------------------------------------------------------

%	\begin{figure}[H]
%		\centering
%		\includegraphics[scale= ]{nombre de la imagen.jpg}
%		\caption{\label{Img:widgets}el pie de pagina que le quieras 	poner a la imagen}
%	\end{figure}
 
%----------------------------------------------------------------------------------------
%            how to insert a table
%----------------------------------------------------------------------------------------

%	\begin{table}[H]
%		\centering
%		\begin{tabular}{|c|c|c|c|}
%			\hline
%			\centering
%				Altura(h) & Distancia (d) & Elaboracion (e) & Longitud (l) \\
%				($\pm0.5$ mm) & ($\pm0.5$ mm) & ($\pm0.5$ mm) & ($\pm0.5$ mm) \\ \hline
%				 &  &  &  \\ \hline
%				 &  &  &  \\ \hline
%				 &  &  &  \\ \hline
%				 &  &  &  \\ \hline
%				 &  &  &  \\ \hline
%		         &  &  &  \\ \hline
%		\end{tabular}
%		\caption{\label{Tab:widgets}pie de pagina que le quieras poner}
%	\end{table}

%----------------------------------------------------------------------------------------
%             How to remove the label in equactions
%----------------------------------------------------------------------------------------

%	\begin{equation*}
%		
%	\end{equation*}

%----------------------------------------------------------------------------------------
%              How to set bibliography
%----------------------------------------------------------------------------------------

%\bibliographystyle{unsrt}
%\bibliography{biblio}
%
%Then you have to set a .bib document such as the next template
%
%	@book{nickname,
%	author = {},
%	title = {},
%	edition = {},
%	year = {},
%	volume = {},
%	ISBN = {}
%	}
%
%	@ARTICLE{nickname,
%	author = {},
%	title = {},
%	year = {},
%	volume = {},
%	}


%----------------------------------------------------------------------------------------
%              END
%----------------------------------------------------------------------------------------